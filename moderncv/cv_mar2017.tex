%% start of file `template_en.tex'.
%% Copyright 2007 Xavier Danaux (xdanaux@gmail.com).
%
% This work may be distributed and/or modified under the
% conditions of the LaTeX Project Public License version 1.3c,
% available at http://www.latex-project.org/lppl/.


\documentclass[11pt]{moderncv}

% moderncv themes
\moderncvtheme[blue]{casual}                 % optional argument are 'blue' (default), 'orange', 'red', 'green', 'grey' and 'roman' (for roman fonts, instead of sans serif fonts)

% character encoding
%\usepackage[utf8]{inputenc}                   % replace by the encoding you are using

% adjust the page margins
\usepackage[scale=0.85]{geometry}
\usepackage[usenames,dvipsnames]{color}
\usepackage{hyperref}
\hypersetup{
  colorlinks,
  citecolor=Violet,
  linkcolor=Red,
  urlcolor=Blue}
\recomputelengths                             % required when changes are made to page layout lengths
\usepackage{ifpdf}

% personal data
\firstname{Joshua}
\familyname{Grant}
%\title{Curriculum Vitae}               % optional, remove the line if not wanted
\address{29 Symons St}{Toronto, Ontario M8V 1T7}    % optional, remove the line if not wanted
\mobile{Mobile: (647) 240-0961}                      % optional, remove the line if not wanted
\email{joshua.m.grant@gmail.com}                      % optional, remove the line if not wanted
\email{Twitter: @joshin4colours}
%\extrainfo{additional information (optional)} % optional, remove the line if not wanted
%\photo[64pt]{my_photo_smaller}                         % '64pt' is the height the picture must be resized to and 'picture' is the name of the picture file; optional, remove the line if not wanted
%\quote{"A mathematician is a machine for turning coffee into theorems." -- Alfr\'ed R\'enyi}                 % optional, remove the line if not wanted

\nopagenumbers{}                             % uncomment to suppress automatic page numbering for CVs longer than one page


%----------------------------------------------------------------------------------
%            content
%----------------------------------------------------------------------------------
\begin{document}
\maketitle

\section{Education}
\cventry{2007-2009}{MSc - Applied Mathematics}{The University of Western Ontario}{London}{Ontario}{Research area: Mathematical Biology}  % arguments 3 to 6 are optional
\cventry{2003-2007}{BSc - Mathematics}{Trent University}{Peterborough}{Ontario}{Dean's Honour List, 2003-2007}  % arguments 

\section{Experience}
\cventry{November 2016--March 2017}{Trading Analyst}{MediaNet}{Toronto}{Ontario}{I was responsible for campaign optimization and peformance, actively looking for optimizations based on data-driven and market-driven approaches. I also implemented some tooling and automation around processes within the advertisting operations team.}
\cventry{December 2015--Ongoing}{Contributing Author}{Techwell Publications}{Toronto}{Ontario}{I've written two articles as a paid contributor to Sticky Minds, a software testing publication. My latest article was \href{https://www.stickyminds.com/article/what-software-testers-need-know-about-automation}{"What Software Testers Need to Know about Automation"}, published TechWell's StickyMinds publication in November 2016.}
\cventry{April 2014}{Webinar Contributor}{SmartBear Software}{Toronto}{Ontario}{I gave a webinar entitled \href{http://blog.smartbear.com/testing/test-design-qa-with-joshin4colours/}{"Automated Test Design: Single Use vs Reusable Tests"} sponsored by SmartBear. This webinar discussed some differences between one-time use automated tools versus tools used often, focusing on design and maintainability.}
\cventry{October 2010--October 2016}{Test Developer}{CaseWare International}{Toronto}{Ontario}{I wrote and maintained automated regression and acceptance tests in both Java and JavaScript (NodeJS), emphasizing maintainability in my test code. I also took an active interest in process automation involving continuous integration services and version control.}
\cventry{August 2009--August 2010}{Research Associate}{Cerebral Diagnostics Canada, Inc}{Toronto}{Ontario}{I was a scientific and technical researcher working for this small biotechnology start-up. We focused on several projects, including developing in-house software for research in EEG (Electroencephalography) and conducting EEG-based studies in pain research.}                % arguments 3 to 6 are optional
\cventry{Summer 2006}{Research Assistant, Department of Mathematics}{Trent University}{Peterborough}{Ontario}{I designed and developed a database of corporate ownership based on SEC filings using Perl and CGI. This was also my first introduction to using regular expressions which I made heavy use of in this project.}                % arguments 3 to 6 are optional
%\subsection{Miscellaneous}
%\cventry{year--year}{Job title}{Employer}{City}{}{Description line 1\newline{}Description line 2}% arguments 3 to 6 are optional

\section{What I'm Looking For:}
\cvlistitem{Opportunities involving working on development-facing tooling and automation}
\cvlistitem{Client-side test development, including end-to-end automation with Selenium and related tools}
\cvlistitem{Server-side test development, including API and performance testing}
\cvlistitem{Data-driven roles and teams making use of modern software methodologies}
\cvlistitem{Roles encouraging collaboration between teams}
\cvlistitem{Roles providing opportunities to improve and expand my programming knowledge, with new or familiar languages and tools}

\newpage

\section{Tools I've Used} 
\cvlistitem{Languages - Java, Python, JavaScript with NodeJS}
\cvlistitem{IDEs - Eclipse, WebStorm, Visual Studio and VS Code}
\cvlistitem{Version Control - Git with Bitbucket, Bitbucket Server and GitHub and a strong interest in working with other systems (SVN, Hg)}
\cvlistitem{Automation Tooling  - Jenkins CI, TravisCI, TestNG, pytest, bash scripting}
%\cvcomputer{}{} {}{}

\section{Interests}
\cvline{Cooking and Dining Out}{This is really my love of food. Toronto is a great food city and I like to make the most of that.}
\cvline{Open Source}{I like seeing what people can create. I've particiated in existing open source projects (\href{www.proselint.com}{www.proselint.com}) and created my own (\href{https://github.com/joshmgrant/corsica}{https://github.com/joshmgrant/corsica})}
\cvline{Writing}{I've always enjoyed writing, both professionally or otherwise. My blog is called Simply the Test (\href{simplythetest.tumblr.com}{simplythetest.tumblr.com})}

\section{References}
\cventry{Victor Amorim}{Vice President}{Account Strategy and Activation}{MediaNet}{(647) 338-5908}{victor.amorim@medianet.com}
\cventry{Brendan Looker}{Team Lead, Web}{Test Developmet}{CaseWare International}{(416) 867-9504 x1184}{brendan.looker@caseware.com}
\cventry{Dr Lindi Wahl}{Professor}{Department of Applied Mathematics}{The University of Western Ontario}{(519) 661-2111 ext 88795}{lwahl@uwo.ca}
\cventry{Dr Geoff Wild}{Professor}{Department of Applied Mathematics}{The University of Western Ontario}{(519) 661-2111 ext 88784}{gwild@uwo.ca} 
%\cventry{Dr Albert Ler}{Research Associate}{Cerebral Diagnostics Canada Inc}{alersh@gmail.com}
%\cventry{$\cdot$}{Dr Marcus Pivato}{Professor}{Department of Mathematics}{Trent University}{(705) 748-1011 ext 7293}{pivato@xaravve.trentu.ca}

%\section{Extra 1}
%\cvlistitem{Item 1}
%\cvlistitem{Item 2}
%\cvlistitem[+]{Item 3}            % optional other symbol

%\section{Extra 2}
%\cvlistdoubleitem[\Neutral]{Item 1}{Item 4}
%\cvlistdoubleitem[\Neutral]{Item 2}{Item 5}
%cvlistdoubleitem[\Neutral]{Item 3}{}

% Publications from a BibTeX file
%\nocite{*}
%\bibliographystyle{plain}
%\bibliography{publications}       % 'publications' is the name of a BibTeX file
\end{document}


%% end of file `template_en.tex'.
