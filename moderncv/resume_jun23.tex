%% start of file `template_en.tex'.
%% Copyright 2007 Xavier Danaux (xdanaux@gmail.com).
%
% This work may be distributed and/or modified under the
% conditions of the LaTeX Project Public License version 1.3c,
% available at http://www.latex-project.org/lppl/.


\documentclass[11pt]{moderncv}

% moderncv themes
\moderncvtheme[blue]{casual}                 % optional argument are 'blue' (default), 'orange', 'red', 'green', 'grey' and 'roman' (for roman fonts, instead of sans serif fonts)
% \moderncvtheme[green]{casual}                % idem

% character encoding
%\usepackage[utf8]{inputenc}                   % replace by the encoding you are using

% adjust the page margins
\usepackage[scale=0.85]{geometry}
\usepackage[usenames,dvipsnames]{color}
\usepackage{hyperref}
\hypersetup{
  colorlinks,
  citecolor=Violet,
  linkcolor=Red,
  urlcolor=Blue}
\recomputelengths                             % required when changes are made to page layout lengths
\usepackage{ifpdf}

% personal data
\firstname{Joshua}
\familyname{Grant}
%\title{Curriculum Vitae}               % optional, remove the line if not wanted
\address{29 Symons St}{Toronto, Ontario M8V 1T7}    % optional, remove the line if not wanted
\mobile{Mobile: (647) 240-0961}                      % optional, remove the line if not wanted
\email{joshua.m.grant@gmail.com}                      % optional, remove the line if not wanted
\email{https://joshgrant.online}
%\extrainfo{additional information (optional)} % optional, remove the line if not wanted
%\photo[64pt]{my_photo_smaller}                         % '64pt' is the height the picture must be resized to and 'picture' is the name of the picture file; optional, remove the line if not wanted
%\quote{"A mathematician is a machine for turning coffee into theorems." -- Alfr\'ed R\'enyi}                 % optional, remove the line if not wanted

\nopagenumbers{}                             % uncomment to suppress automatic page numbering for CVs longer than one page


%----------------------------------------------------------------------------------
%            content
%----------------------------------------------------------------------------------
\begin{document}
\maketitle

\section{Relevant Experience}
\cventry{October 2022--May 2023}{Developer Advocate}{Code Intelligence}{Bonn/Remote}{Germany/Canada}{As a developer advocate at an early-stage startup, I focused on content engineering. This included writen and video content based on fuzzers generally and Code Intelligence's main product CIFuzz in particular.}                % arguments 3 to 6 are optional
\cventry{March 2018--August 2022}{Solution Architect}{Sauce Labs}{San Fransisco/Remote}{USA/Canada}{In this customer-facing role, I provided post-sale technical guidance and advice to enteprise customers working with Sauce Labs. This included developing best practices for Sauce products, troubleshooting test automation issues and demoing new and existing features to existing customers. My goal was to create awareness of fuzzing and CIFuzz in particular.}                % arguments 3 to 6 are optional
\cventry{April 2017--February 2018}{Test Developer}{JUICE Mobile}{Toronto}{Ontario}{I worked with a programmatic advertisting platform, primarily developing a service-level or API test framework in Python. I worked on a team of 10 developers as an "embedded" test development specialist.}                % arguments 3 to 6 are optional
\cventry{October 2010--October 2016}{Test Developer}{CaseWare International}{Toronto}{Ontario}{I worked primarily with the CaseWare Time product. I wrote and maintained automated regression and acceptance tests, emphasizing maintainability in my test code. This included working with Java-based Selenium tests for Time on the Go, Time's web-based version and SilkTest suites for Time, the primary desktop product.}                % arguments 3 to 6 are optional
%\subsection{Miscellaneous}
%\cventry{year--year}{Job title}{Employer}{City}{}{Description line 1\newline{}Description line 2}% arguments 3 to 6 are optional

\section{Education}
\cventry{2007-2009}{MSc - Applied Mathematics}{The University of Western Ontario}{London}{Ontario}{Research area: Mathematical Biology}  % arguments 3 to 6 are optional
\cventry{2003-2007}{BSc - Mathematics}{Trent University}{Peterborough}{Ontario}{Dean's Honour List, 2003-2007}  % arguments 

\section{What I'm Looking For:}
\cvlistitem{Content Engineering: If you need a public speaker or writer for your technical content, I'm happy to help}
\cvlistitem{Test Automation and Architecture: I've worked at all levels of the test automation pyramid and would like to continue to do so}
\cvlistitem{Python: for web development, test automation and more, I've been working with Python professionally for 8+ years}
%\section{Strengths and Skills}
%\cvlistitem{Strong personal and technical communication skills}
%\cvlistitem{Ability to interact and work in variety of settings and with various personalities}
%\cvlistitem{Excellent problem solving skills}
%\cvlistitem{Strengths in project planning and abstracting reasoning}
%\cvlistitem{Experience with individual-based and Monte Carlo simulation}
%\cvlistitem{Knowledge and experience in object-oriented programming}
%\cvlistitem{Experience with statistical and data analysis in non-academic settings, including non-parametric statistics and stochastic processes}

\section{Tools I Love} 
\cvlistitem{Python, for test automation and web development}
\cvlistitem{JetBrains, for Java development in IntelliJ}
\cvlistitem{Camtasia, for creating captivating video content}
\cvlistitem{Taking short walks along the lake, for excellent debugging}

\newpage

\section{What I've Done, in Chronological Order}
\cvlistitem{Built and maintained a medium-large Java-based test framework for a project management web application of approximately 1000 tests. This framework used Selenium with TestNG and achieved partial parallelization.}
\cvlistitem{Built a unit test suite for an open source project called Proselint, writing approximately 50 test functions in Pytest.}
\cvlistitem{Built an API test framework for a programmatic ad platform product using Python. This tested the API layer of the internal application using approximately 200 tests with Pytest and requests.}
\cvlistitem{Designed and delivered a standard presentation on best practices in Continous Integration/Continous Deployment. This presentation was delivered as part of solution architech work to enterprise company customers such as Accenture and at tech conferences such as Targeting Quality by KWSQA.}
\cvlistitem{Designed and delivered a presentation called "What the Fuzz? An Introduction to Fuzz Testing". This talk was delivered at multiple meetups including PyLadies Toronto and BSides Rochester.}

\section{Interests}
\cvline{Food}{I enjoy eating and cooking food of all kinds. Currently living in Toronto which an excellent food city overall.}
\cvline{Art}{Current interests include work by Agnes Martin and Louis Kahn.}
\cvline{Being Online}{I am quite active on social media, including Twitter \emph{@joshin4colours} and Mastodon \emph{mastodon.social/@joshin4colours}}

% \section{References}
% \cventry{Khaled Yakdan}{Co-Founder and Chief Scientist}{Code Intelligence}{yakdan@code-intelligence.com}
% \cventry{Evelyn Coleman}{Manager, Implementation Engineering}{Sauce Labs}{coleve27@gmail.com}
% \cventry{Marion Nehring}{Director, Developer Relations}{Code Intelligence}{marion.nehring@outlook.com} 
% \cventry{Nikolay Advolodkin}{Senior Developer Advocate}{Sauce Labs}{nadvolod@gmail.com}
%\cventry{$\cdot$}{Dr Marcus Pivato}{Professor}{Department of Mathematics}{Trent University}{(705) 748-1011 ext 7293}{pivato@xaravve.trentu.ca}

%\section{Extra 1}
%\cvlistitem{Item 1}
%\cvlistitem{Item 2}
%\cvlistitem[+]{Item 3}            % optional other symbol

%\section{Extra 2}
%\cvlistdoubleitem[\Neutral]{Item 1}{Item 4}
%\cvlistdoubleitem[\Neutral]{Item 2}{Item 5}
%cvlistdoubleitem[\Neutral]{Item 3}{}

% Publications from a BibTeX file
%\nocite{*}
%\bibliographystyle{plain}
%\bibliography{publications}       % 'publications' is the name of a BibTeX file
\end{document}


%% end of file `template_en.tex'.
